% Use the content.tex file to enter your content.

\documentclass[sigconf, nonacm, screen=True, review=False, anonymous=False, authordraft=False]{acmart}

%%% START ADDITIONAL PACKAGES AND COMMANDS %%% 
\renewcommand{\sectionautorefname}{Section}
\renewcommand{\subsectionautorefname}{Section}
\renewcommand{\subsubsectionautorefname}{Section}

\usepackage[yyyymmdd]{datetime}
\renewcommand{\dateseparator}{--}

\usepackage{tabularx}
\usepackage{dcolumn} %Aligning numbers by decimal points in table columns
\newcolumntype{d}[1]{D{.}{.}{#1}}

\usepackage{subcaption}

\usepackage{todonotes}
\let\xtodo\todo
\renewcommand{\todo}[1]{\xtodo[inline,color=green!50]{#1}}
\newcommand{\itodo}[1]{\xtodo[inline]{#1}}
\newcommand{\red}[1]{\textcolor{red}{#1}}
%%% END ADDITIONAL PACKAGES AND COMMANDS %%% 


\newcommand{\charactercount}[1]{
\immediate\write18{
    expr `texcount -1 -sum -merge #1.tex` + `texcount -1 -sum -merge -char #1.tex` - 1 
    > chars.txt
}\input{chars.txt}}


%%
%% \BibTeX command to typeset BibTeX logo in the docs
\AtBeginDocument{%
  \providecommand\BibTeX{{%
    Bib\TeX}}}

% You may change the licence of our work.
\setcopyright{CC}
\setcctype{by-sa}


\begin{document}

%%
%% The "title" command has an optional parameter,
%% allowing the author to define a "short title" to be used in page headers.
\title{This is a Human-Computer Interaction Seminar Paper}
%\title[The Short Title]{This is a Human-Computer Interaction Seminar Paper}

\author{Your Name}
\orcid{0000-0000-0000-0000} % get your own ORCID via https://orcid.org/
\affiliation{%
  \institution{LMU Munich}
  \city{Munich}
  \postcode{80337}
  \country{Germany}}
\email{your.name@campus.lmu.de}

%%
%% The abstract is a short summary of the work to be presented in the
%% article.
\begin{abstract}
% Enter your abstract here, we advise following the structure given:
\todo{What is the specific problem addressed?}
\todo{What have you done?}
\todo{What did you find out?}
\todo{What are the implications on a larger scale?}
\todo{Note: aim for 150 words}
\end{abstract}

%%
%% The code below is generated by the tool at http://dl.acm.org/ccs.cfm.
%% Please copy and paste the code instead of the example below.
%%
\begin{CCSXML}
<ccs2012>
    <concept>
        <concept_id>10003120.10003121.10003128</concept_id>
        <concept_desc>Human-centered computing~Human computer interaction (HCI)</concept_desc>
        <concept_significance>300</concept_significance>
    </concept>
 </ccs2012>
\end{CCSXML}
\ccsdesc[500]{Human-centered computing~Human computer interaction (HCI)}


%%
%% Keywords. The author(s) should pick words that accurately describe
%% the work being presented. Separate the keywords with commas.
\keywords{human computer interaction}


%% A "teaser" image appears between the author and affiliation
%% information and the body of the document, and typically spans the
%% page.
%\begin{teaserfigure}
%  \includegraphics[width=\linewidth]{sampleteaser}
%  \caption{Seattle Mariners at Spring Training, 2010.}
%  \Description{Enjoying the baseball game from the third-base
%  seats. Ichiro Suzuki preparing to bat.}
%  \label{fig:teaser}
%\end{teaserfigure}



\maketitle

\todo{The section structure is a proposal but might need to be adjusted depending on your specific needs.}

% Use this file to enter your text.
\section{Introduction}

% First Paragraph
% CORE MESSAGE OF THIS PARAGRAPH:
\todo{P1.1. What is the large scope of the problem?}
\todo{P1.2. What is the specific problem?}

% Second Paragraph
% CORE MESSAGE OF THIS PARAGRAPH:
\todo{P2. The second paragraph should be about what have others been doing}
\todo{P2.3. Why is the problem important? Why was this work carried out?}

% Third Paragraph
% CORE MESSAGE OF THIS ddsfsfPARAGRAPH:
\todo{P3.4. What have you done?}
\todo{P3.5. What is new about your work?}

% Fourth paragraph
% CORE MESSAGE OF THIS PARAGRAPH:
\todo{P4.6. What did you find out? What are the concrete results?}
\todo{P4.7. What are the  implications? What does this msfasasean for the bigger picture?}

\section{Methodology}

\section{Results}

\section{Discussion}

\section{Conclusion}


%TC:ignore
\bibliographystyle{ACM-Reference-Format}
\bibliography{bibliography}

\appendix
\section{Character Count}
This work contains \charactercount{content} characters.
%TC:endignore
\end{document}